 \documentclass[12pt]{article}
\usepackage[T2A]{fontenc}
\usepackage[utf8]{inputenc}        % Кодировка входного документа;
                                    % при необходимости, вместо cp1251
                                    % можно указать cp866 (Alt-кодировка
                                    % DOS) или koi8-r.

\usepackage[english,russian]{babel} % Включение русификации, русских и
                                    % английских стилей и переносов
%%\usepackage{a4}
%%\usepackage{moreverb}
\usepackage{amsmath,amsfonts,amsthm,amssymb,amsbsy,amstext,amscd,amsxtra,multicol}
\usepackage{verbatim}
\usepackage{mathtools}
\usepackage{tikz} %Рисование автоматов
\usetikzlibrary{automata,positioning}
\usepackage{multicol} %Несколько колонок
\usepackage{graphicx}
\usepackage{ulem}
\usepackage[colorlinks,urlcolor=blue]{hyperref}
\usepackage[stable]{footmisc}
%\usepackage{hyperref}

%% \voffset-5mm
%% \def\baselinestretch{1.44}
\renewcommand{\theequation}{\arabic{equation}}
\def\hm#1{#1\nobreak\discretionary{}{\hbox{$#1$}}{}}
\newtheorem{Lemma}{Лемма}
\theoremstyle{definiton}
%%\newtheorem{Def}{Определение}
\newtheorem{Claim}{Утверждение}
\newtheorem{Cor}{Следствие}
\newtheorem{Theorem}{Теорема}
\theoremstyle{definition}
\newtheorem{Example}{Пример}
\newtheorem*{known}{Теорема}
\def\proofname{Доказательство}
\def\solutionname{Решение}
\theoremstyle{definition}
\newtheorem{Def}{Определение}

%% \newenvironment{Example} % имя окружения
%% {\par\noindent{\bf Пример.}} % команды для \begin
%% {\hfill$\scriptstyle\qed$} % команды для \end






%\date{22 июня 2011 г.}
\let\leq\leqslant
\let\geq\geqslant
\def\MT{\mathrm{MT}}
%Обозначения ``ажуром''
\def\BB{\mathbb B}
\def\CC{\mathbb C}
\def\RR{\mathbb R}
\def\SS{\mathbb S}
\def\ZZ{\mathbb Z}
\def\NN{\mathbb N}
\def\FF{\mathbb F}
%греческие буквы
\let\epsilon\varepsilon
\let\es\emptyset
\let\eps\varepsilon
\let\al\alpha
\let\sg\sigma
\let\ga\gamma
\let\ph\varphi
\let\om\omega
\let\ld\lambda
\let\Ld\Lambda
\let\vk\varkappa
\let\Om\Omega
\def\abstractname{}

\def\R{{\cal R}}
\def\A{{\cal A}}
\def\B{{\cal B}}
\def\C{{\cal C}}
\def\D{{\cal D}}
\let\w\omega


%вероятность 
\newcommand{\Expect}{\mathsf{E}}
\newcommand{\MExpect}{\mathsf{M}}
\newcommand{\Disp}{\mathsf{D}}

%классы сложности
\def\REG{{\mathsf{REG}}}
\def\CFL{{\mathsf{CFL}}}
\newcounter{problem}
\newcounter{uproblem}
\newcounter{subproblem}
\def\pr{\medskip\noindent\stepcounter{problem}{\bf \theproblem .  }\setcounter{subproblem}{0} }
\def\prstar{\medskip\noindent\stepcounter{problem}{\bf $\theproblem^*$\negthickspace.  }\setcounter{subproblem}{0} }
\def\prpfrom[#1]{\medskip\noindent\stepcounter{problem}{\bf Задача \theproblem~(№#1 из задания).  }\setcounter{subproblem}{0} }
\def\prp{\medskip\noindent\stepcounter{problem}{\bf Задача \theproblem .  }\setcounter{subproblem}{0} }
\def\prpstar{\medskip\noindent\stepcounter{problem}{\bf Задача $\bf\theproblem^*$\negthickspace.  }\setcounter{subproblem}{0} }
\def\prdag{\medskip\noindent\stepcounter{problem}{\bf Задача $\theproblem^{^\dagger}$\negthickspace\,.  }\setcounter{subproblem}{0} }
\def\upr{\medskip\noindent\stepcounter{uproblem}{\bf Упражнение \theuproblem .  }\setcounter{subproblem}{0} }
%\def\prp{\vspace{5pt}\stepcounter{problem}{\bf Задача \theproblem .  } }
%\def\prs{\vspace{5pt}\stepcounter{problem}{\bf \theproblem .*   }
\def\prsub{\medskip\noindent\stepcounter{subproblem}{\rm \thesubproblem .  } }
\def\prsubstar{\medskip\noindent\stepcounter{subproblem}{\rm $\thesubproblem^*$\negthickspace.  } }
%прочее
\def\quotient{\backslash\negthickspace\sim}
\begin{document}

\centerline{\LARGE Домашнее задание 7, Новиков Герман, 277}

\medskip

%Все вычисления вместе с кодом и комментариями находятся в файле на $\href{https://github.com/elejke/statistics_homework/blob/master/task3/code/task3.ipynb}{\textbf{github}}$

\prp

\prp

\prp (Задача 19). При измерении длины стержня, истинная длина которого
равна $l > 0$ (и неизвестна), ошибка измерения имеет распределение $N(0, kl)$, где $k$ — известное число. Найти оценку наибольшего правдоподобия для параметра $l$, построенную на основании независимых измерений $X_1, X_2, . . . , X_n$ длины стержня.

\textbf{Решение: } Строим оценку максимума правдоподобия. Выпишем правдоподобие:

$$\mathcal{L}(l \,;\,X_1,\ldots,X_n) = f(X_1,X_2,\ldots,X_n\mid l) = \prod_{i=1}^n f(X_i\mid l) = \prod_{i=1}^n \frac{1}{\sqrt{2\pi kl}} \exp{ \left(-\frac{(X_i - l)^2}{2kl}\right)}$$

Логарифмируем (ОМП для логарифма правдоподобия будет совпадать с первоначаналньым).

$$\mathcal{\log{L}} = -\frac{n}{2}\log{l} + \sum_{i=1}^{n} \left(- \frac{(X_i-l)^2}{2kl}\right) = -\frac{n}{2}\log{l} + \sum_{i=1}^{n} \left(- \frac{(X_i^2 - 2X_i l + l^2)}{2kl}\right)$$

Максимизируем логарифм правдоподобия:

$$0 = \frac{\partial}{\partial l}\mathcal{\log{L}} = -\frac{n}{2l} + \sum_{i=1}^{n} \frac{X_i^2}{2kl^2} - n \frac{1}{2k}$$

$$ n l^2 + n k l - \sum_{i=1}^{n} X_i^2 = 0$$

И находим окончательно - ОМП для параметра $l$:

$$ \hat l =  \frac{- k + \sqrt{(k^2 + \frac{4}{n} \sum_{i=1}^{n} X_i^2)}}{2}$$


%$$\hat l =  \frac{1}{k}\sqrt{\frac{1}{n}\sum_{i=1}^{n} X_i^2}$$



\prp (Задача 20) Найти оценки наибольшего правдоподобия и эффективные оценки (если они существуют): а) параметра $\lambda$ в пуассоновском распределении б) параметра $\mu$ в показательном распределении в) параметра $p$ в биномиальном распределении с $n$ испытаниями. Являются ли полученные оценки несмещенными, состоятельными?

\textbf{Решение: } 

a) Строим ОМП для $\lambda$ в пуассоновском распределении:

Функция правдоподобия:

$$\mathcal{L}(\lambda \,;\,X_1,\ldots,X_n) = \prod\limits_{i=1}^n \frac{\lambda^{X_i} \exp{(-\lambda)}}{X_i!}$$

Логарифм правдоподобия:

$$l = \log \mathcal{L}(\lambda \,;\,X_1,\ldots,X_n)= \sum\limits_{i=1}^n X_i\log{\lambda} - n \lambda - \sum\limits_{i=1}^n \log{(X_i!)}$$

Максимизируем логарифм правдоподобия:

$$0 = \frac{\partial}{\partial \lambda} l = \frac{1}{\lambda} \sum\limits_{i=1}^n X_i - n$$

Таким образом:

$$\hat \lambda = \frac{\sum\limits_{i=1}^n X_i}{n}$$

Проверяем: 

i) Несмещенность:

$$\Expect \hat \lambda = \Expect \frac{\sum\limits_{i=1}^n X_i}{n} = \frac{\sum\limits_{i=1}^n \Expect X_i}{n} = \lambda \rightarrow unbiased$$

ii) Состоятельность:

Для состоятельности необходимо, что бы предел $$ \forall \epsilon > 0 \lim_{n \to \infty} \Prob \{ |\hat\lambda - \lambda| < \epsilon \} = 1 $$

Доказательство по ЗБЧ: условия одинаковой распределенности, существования мат.ожиданий и убывающих дисперсий:

$$\mathsf{D}\hat\lambda = \frac{1}{n^2}\sum\limits_{i=1}^n \mathsf{D} X_i = \frac{1}{n^2} n \lambda = \frac{\lambda}{n} = O\left(\frac{1}{n}\right), n \to \infty$$

Таким образом, это состоятельная оценка.

iii) Эффективная оценка: 

Воспользуемся следствием из неравенства Рао-Крамера:

(R) Для любого $y \in \mathbb{R_+}$ выполнена дифференцируемость $\sqrt{f_{\lambda}(y)}$ по параметру $\lambda$

(RR) Информация Фишера существует, положительна и непрерывна во всех точках.

\textbf{Следствие :} Если в неравенстве Рао-Крамера достигается равенство, то оценка $\hat \theta$ эфеективна в классе несмещенных (в нашем частном случае).

Для дважды дифф. по $\lambda$:

$$I_n(\lambda) = -  \Expect \left(\frac{\partial^2 l}{\partial \lambda^2}\right) = \Expect \frac{1}{\lambda^2} \sum\limits_{i=1}^n X_i = \frac{1}{\lambda^2} \sum\limits_{i=1}^n \Expect X_i = \frac{n}{\lambda}$$


Но тогда и получаем необходимое равенство:

$$\mathsf{D}\hat\lambda = \frac{\lambda}{n} = \frac{1}{I_n(\lambda)}$$

То есть, ОМП $\hat\lambda$ - \textbf{эффективная}.

\bigskip

б)  Строим ОМП для параметра $\mu$ в показательном распределении:

$$\mathcal{L}(\mu \,;\,X_1,\ldots,X_n) = \prod\limits_{i=1}^n \mu \exp{(-\mu X_i)}$$

Логарифм правдоподобия:

$$l = \log \mathcal{L}(\mu \,;\,X_1,\ldots,X_n) = n \log (\mu) - \sum\limits_{i=1}^n \mu X_i  $$

Минимизируем логарифм правдоподобия:

$$0 = \frac{\partial}{\partial \mu} l = \frac{n}{\mu} - \sum\limits_{i=1}^n X_i$$

Получаем ОМП для $\mu$:

$$\hat \mu = \frac{n}{\sum\limits_{i=1}^n X_i}$$


Проверяем: 

i) Несмещенность:

$$\Expect \hat \mu  = \Expect \frac{n}{\sum\limits_{i=1}^n X_i}$$

Распределение суммы показательно распределенных величин через гамма-распределение:

$$\Prob_{\sum X_i} (x) = \mathbb{I}(X \geq 0) \frac{\mu^n x^{n-1}}{\Gamma(n)}$$


$$\Expect \frac{1}{\sum X_i} = \int\limits_0^{\infty} \frac{1}{x} \frac{\mu^n}{\Gamma(n)} x^{n-1} \exp{(-\mu x)} dx = \frac{\mu}{n-1}$$

$$\Expect \hat \mu = n \Expect \frac{1}{\sum X_i} = n \frac{\mu}{n-1} = \frac{n}{n-1} \mu$$

То есть замечаем, что не является несмещенной, однако, является ассимтотически несмещенной.


ii) Состоятельность:

$$\Disp \hat\mu = n^2 \Disp \left(\frac{1}{\sum X_i} \right) = n^2 \left( \Expect  \left(\frac{1}{\sum X_i} \right)^2 -  \left(\Expect\frac{1}{\sum X_i}\right)^2 \right) $$

Посчитаем только $$\Expect  \left(\frac{1}{\sum X_i} \right)^2 = $$

Аналогично матожиданию, с помощью гамма-функции:

$$= \frac{mu^2}{\Gamma(n)} \Gamma(n-2) = \frac{\mu^2}{(n-1)(n-2)}$$

И для дисперсии:

$$\Disp \hat\mu = \frac{n^2 \mu^2}{(n-1)^2(n-2)} = O\left(\frac1n\right), n \to \infty$$

То есть, аналогично пункту (а) получаем \textbf{состоятельность} оценки.

iii) Эффективность:

$$I_n(\mu) = -  \Expect \left(\frac{\partial^2 l}{\partial \mu^2}\right) = -\Expect \left( - \frac{n}{\mu^2}\right) = \frac{n}{\mu^2} \neq  \frac{n^2 \mu^2}{(n-1)^2(n-2)} = \Disp \hat\mu$$

То есть не является эффективной, однако, при $n \to \infty$ равенство выполнено:

$$\lim_{n \to \infty} \Disp \hat\mu = \lim_{n \to \infty} \frac{n^2 \mu^2}{(n-1)^2(n-2)} =  \frac{\mu^2}{n} = \frac{1}{I_n(\mu)}$$

То есть -\textbf{ ассимптотически эффективная}.

\bigskip

в) Строим ОМП для параметра $p$ в биномиальном распределении:

$$\mathcal{L}(p \,;\,X_1,\ldots,X_n) = \prod\limits_{i=1}^n p^{X_i} (1-p)^{(1 - X_i)} = p^{\sum\limits_{i=1}^n X_i} (1-p)^{n - \sum\limits_{i=1}^n X_i}$$

Логарифм правдоподобия:

$$l = \log \mathcal{L}(p \,;\,X_1,\ldots,X_n) =  \log{p} \sum\limits_{i=1}^n X_i + \log{(1-p)} (n - \sum\limits_{i=1}^n X_i) $$

Максимизируем логарифм правдоподобия:

$$0 = \frac{\partial}{\partial p} l = \frac1p \sum\limits_{i=1}^n X_i - \frac{1}{(1-p)} (n - \sum\limits_{i=1}^n X_i) = $$

$$  = 1 \sum\limits_{i=1}^n X_i  - p \sum\limits_{i=1}^n X_i  - p n + p \sum\limits_{i=1}^n X_i$$

И находим ОМП для $p$:

$$\hat p = \frac{ \sum\limits_{i=1}^n X_i}{n}$$ 

Проверяем: 

i) Несмещенность:

$$\Expect \hat p  = \Expect \frac{ \sum\limits_{i=1}^n X_i}{n} =  \frac{ \sum\limits_{i=1}^n \Expect X_i}{n} = \frac{np}{n} = p$$

ii) Состоятельность: Аналогично пункту (а) проверяем убывание к нулю дисперсий (все остальные условия выполнены):

$$\mathsf{D} \hat p = \mathsf{D} \left(\frac{ \sum\limits_{i=1}^n X_i}{n}\right) = \frac{1}{n^2} \mathsf{X_i} = \frac{1}{n} p(1-p) = O\left( \frac1n \right), n \to \infty$$

То есть оценка \textbf{состоятельная.}

iii) Эффективная:

Аналогично - используем следствие из неравенства Рао-Крамера:


$$I_n(p) = -  \Expect \left(\frac{\partial^2 l}{\partial p^2}\right) =  - \Expect \left( - \frac{1}{p^2} \sum X_i - \frac{1}{(1-p)^2} (n - \sum X_i) \right) = -\frac{n}{p} + \frac{n}{(p-1)} = \frac{n}{p(p-1)}$$


Получаем равенство в неравенстве:

$$\Disp \hat{p}^2 = \frac{p(p-1)}{n} =  \frac{1}{I_n(p)}$$

То есть являтеся \textbf{эффективной}. 


\prp (Задача 40). В результате наблюдения точечного случайного процесса (потока событий) получена выборка $(X_1, . . . , X_n)$ моментов появления в нем событий. Предполагая, что наблюдаемый процесс является пуассоновским, найти ММП-оценки для интервала времени между событиями и для интенсивности потока событий.

\textbf{Решение: } ФР интервалов $T = X_{j+1} - X_j, j = \overline{1,n-1}$:

$$F_T(t) = \Prob(T \leq t) = 1 - \exp{(-\lambda t)}$$

Плотность: $f_T(t) = \lambda \exp{(-\lambda t)}$

Найдём функцию правдоподобия:


$$\mathcal{L}(\lambda \,;\,X_1,\ldots,X_n) = \prod\limits_{i=1}^n \lambda \exp{(-\lambda X_i)} = \lambda^n \exp{(-\lambda\sum X_i)}$$

Логарифм правдоподобия:

$$l = \log{(\mathcal{L}(\lambda \,;\,X_1,\ldots,X_n))} = n \log \lambda - \lambda \sum X_i$$ 

Максимизируем логарифм правдоподобия:

$$\frac{\partial l}{\partial \lambda} = n \frac{1}{\lambda} - \sum X_i = 0$$

ОМП:

$$ \hat \lambda = \frac{n}{\sum X_i}$$

Теперь, так как выполнено условие непрерывности функции распределения, оценку для $T$ можно получить используя:

$$\hat{T} = \hat{X_{1}} = \hat{\left(\frac{1}{\lambda}\right)} = \frac{1}{\hat\lambda} = \frac{\sum X_i}{n}$$

Так как промежутки между событиями не зависят от других промежутков.

\prp (Задача 64). Значение сигнала $Y$ на входе некоторого устройства может быть либо нулем, либо единицей. Значение $Y$ недоступно для измерения. На выходе устройства наблюдается (и измеряется) величина $X$, являющаяся суммой входного сигнала и гауссовского шума с нулевым математическим ожиданием и известной дисперсией $\sigma^2$.Построить оптимальное байесовское решающее правило для классификации входных сигналов на основании измерения величины $X$ при известных вероятностях $\Prob\{Y = 0\} = p, \Prob\{Y = 1\} = 1 - p$.

\textbf{Решение: } Найдем апостериорные вероятности классов по формуле:

$$\pi(i|x) = \pi_i(x) = \frac{f_i(x) \pi_i}{\sum\limits_{s=0}^1\pi_s f_s(x)}, \, \pi_0 = p, \pi_1 = 1 - p$$

Где плотности имеют вид:

$$f_0(x) = \frac{1}{\sqrt{2\pi \sigma^2}}\exp{\left(-\frac{x^2}{2\sigma^2}\right)}, f_1(x) = \frac{1}{\sqrt{2\pi \sigma^2}}\exp{\left(-\frac{(x-1)^2}{2\sigma^2}\right)}$$

В силу того, что вход и выход бинарные $\{0,1\}$, то линейная функция потерь совпадает с квадратичной и задаётся формулой:

$$l(i|j) = (i-j)^2 = |i-j|$$

В таком случае байесовский принцип сводится к тому, что бы относить объект с характеристикой $x$ к тому классу, апостериорная вероятность которого максимальна (\textit{принцип максимума апостериорной вероятности}):

Забивая на знаменатели, минимизируем только функции:

$$h_j(x) = \sum\limits_{i=1}^k l(j|i) \pi_i f_i(x), \, j=\overline{1,k}$$

В нашем случае:

$$h_0(x) = l(0|1)\pi_1 f_1(x), \, \, h_1(x) = l(1|0) \pi_0 f_0(x)$$

Таким образом, решение отнести объект $x$ к первому классу принимается тогда и только тогда, когда $\pi_1 f_1(x) \leq \pi_0 f_0(x)$/

Оптимальное байесовское правило $\delta^*$ имеет в этом случае вид:

\[
	\delta^* =
	\begin{cases}
		0, & \text{если $x \in W_0^*$} \\
		1, & \text{если $x \in \overline{W_0^*}$}
	\end{cases}
\]

Где $W_1^*$ выражается через вероятности:

\[
W_0^* = \Bigl\{ x : \frac{f_1(x)}{f_0(x)} \leq \frac{\pi_0}{\pi_1}
\Bigr\}
\]

Остается только вычислить эти отношения:

$$\exp{\left(-\frac{(-2x + 1)}{2\sigma^2}\right)} \leq \frac{p}{1-p} $$

$$\frac{2x - 1}{2\sigma^2} \leq  \log {( \frac{p}{1-p} )} $$ 

То есть при $ x \leq \sigma^2 \log {( \frac{p}{1-p} )} + \frac12$ нужно выбирать $0$, а иначе $1$.





\end{document}