 \documentclass[12pt]{article}
\usepackage[T2A]{fontenc}
\usepackage[utf8]{inputenc}        % Кодировка входного документа;
                                    % при необходимости, вместо cp1251
                                    % можно указать cp866 (Alt-кодировка
                                    % DOS) или koi8-r.

\usepackage[english,russian]{babel} % Включение русификации, русских и
                                    % английских стилей и переносов
%%\usepackage{a4}
%%\usepackage{moreverb}
\usepackage{amsmath,amsfonts,amsthm,amssymb,amsbsy,amstext,amscd,amsxtra,multicol}
\usepackage{verbatim}
\usepackage{mathtools}
\usepackage{tikz} %Рисование автоматов
\usetikzlibrary{automata,positioning}
\usepackage{multicol} %Несколько колонок
\usepackage{graphicx}
\usepackage[colorlinks,urlcolor=blue]{hyperref}
\usepackage[stable]{footmisc}
%\usepackage{hyperref}

%% \voffset-5mm
%% \def\baselinestretch{1.44}
\renewcommand{\theequation}{\arabic{equation}}
\def\hm#1{#1\nobreak\discretionary{}{\hbox{$#1$}}{}}
\newtheorem{Lemma}{Лемма}
\theoremstyle{definiton}
%%\newtheorem{Def}{Определение}
\newtheorem{Claim}{Утверждение}
\newtheorem{Cor}{Следствие}
\newtheorem{Theorem}{Теорема}
\theoremstyle{definition}
\newtheorem{Example}{Пример}
\newtheorem*{known}{Теорема}
\def\proofname{Доказательство}
\def\solutionname{Решение}
\theoremstyle{definition}
\newtheorem{Def}{Определение}

%% \newenvironment{Example} % имя окружения
%% {\par\noindent{\bf Пример.}} % команды для \begin
%% {\hfill$\scriptstyle\qed$} % команды для \end






%\date{22 июня 2011 г.}
\let\leq\leqslant
\let\geq\geqslant
\def\MT{\mathrm{MT}}
%Обозначения ``ажуром''
\def\BB{\mathbb B}
\def\CC{\mathbb C}
\def\RR{\mathbb R}
\def\SS{\mathbb S}
\def\ZZ{\mathbb Z}
\def\NN{\mathbb N}
\def\FF{\mathbb F}
%греческие буквы
\let\epsilon\varepsilon
\let\es\emptyset
\let\eps\varepsilon
\let\al\alpha
\let\sg\sigma
\let\ga\gamma
\let\ph\varphi
\let\om\omega
\let\ld\lambda
\let\Ld\Lambda
\let\vk\varkappa
\let\Om\Omega
\def\abstractname{}

\def\R{{\cal R}}
\def\A{{\cal A}}
\def\B{{\cal B}}
\def\C{{\cal C}}
\def\D{{\cal D}}
\let\w\omega


%вероятность 
\newcommand{\Expect}{\mathsf{E}}
\newcommand{\MExpect}{\mathsf{M}}

%классы сложности
\def\REG{{\mathsf{REG}}}
\def\CFL{{\mathsf{CFL}}}
\newcounter{problem}
\newcounter{uproblem}
\newcounter{subproblem}
\def\pr{\medskip\noindent\stepcounter{problem}{\bf \theproblem .  }\setcounter{subproblem}{0} }
\def\prstar{\medskip\noindent\stepcounter{problem}{\bf $\theproblem^*$\negthickspace.  }\setcounter{subproblem}{0} }
\def\prpfrom[#1]{\medskip\noindent\stepcounter{problem}{\bf Задача \theproblem~(№#1 из задания).  }\setcounter{subproblem}{0} }
\def\prp{\medskip\noindent\stepcounter{problem}{\bf Задача \theproblem .  }\setcounter{subproblem}{0} }
\def\prpstar{\medskip\noindent\stepcounter{problem}{\bf Задача $\bf\theproblem^*$\negthickspace.  }\setcounter{subproblem}{0} }
\def\prdag{\medskip\noindent\stepcounter{problem}{\bf Задача $\theproblem^{^\dagger}$\negthickspace\,.  }\setcounter{subproblem}{0} }
\def\upr{\medskip\noindent\stepcounter{uproblem}{\bf Упражнение \theuproblem .  }\setcounter{subproblem}{0} }
%\def\prp{\vspace{5pt}\stepcounter{problem}{\bf Задача \theproblem .  } }
%\def\prs{\vspace{5pt}\stepcounter{problem}{\bf \theproblem .*   }
\def\prsub{\medskip\noindent\stepcounter{subproblem}{\rm \thesubproblem .  } }
\def\prsubstar{\medskip\noindent\stepcounter{subproblem}{\rm $\thesubproblem^*$\negthickspace.  } }
%прочее
\def\quotient{\backslash\negthickspace\sim}
\begin{document}

\centerline{\LARGE Домашнее задание 3, Новиков Герман, 277}

\medskip

Все вычисления вместе с кодом и комментариями находятся в файле на $\href{https://github.com/elejke/statistics_homework/blob/master/task3/code/task3.ipynb}{\textbf{github}}$

\prp 8-го января 2003 года в New York Times были сообщены следующие
данные из штата Мэриленд: в случае, если происходило убийство афроамериканца,
было вынесено 14 смертных приговоров для преступника, а в 641 случае смертных
приговор не последовало. В случае, если происходило убийство белого, то в 62
случаях был вынесен смертный приговор и в 594 случаях не был. Проанализируйте
эти данные, используя статистические техники, и интерпретируйте результаты.


\textbf{Решение: } Проверим, является ли значимым цвет кожи убитого, то есть посмотрим - есть ли зависимость между количеством осужденных за убийство человека того или иного цвета кожи от самого цвета кожи убитого. Выдвинем гипотезу, о том что зависимости нет и воспользуемся критерием $\chi^2$:


Ответ:
%Получаемое значение $T_{\chi^2}$ отвечает нашему предположению для $\alpha = 0.05$, соотвественно, принимаем гипотезу о равномерном распределении.

\prp (Задача номер 59). Построить критерий для проверки гипотезы $H_1 : p = \frac12$ при альтернативной гипотезе $H_2 : p \neq \frac12$ по результатам восьми испытаний, подчиняющихся схеме Бернулли. Вероятность ошибки первого рода $\alpha$ положить равной 0,05.


\textbf{Решение: } Предположим, что верная гипотеза $H_1 : p = \frac12$. В этом предположении случайная величина $$\theta = 2\sqrt{n} (\frac{n-k}{k} - \frac12),$$ где $n = 8$, $k-$ количество $1$, имеет в силу ЦПТ распределение к близкое к $N(0,1)$ (сходимость по вероятности).И, таким образом, можно установить пару квантилей, соответствующих $\frac{\alpha}{2}$ и $1 - \frac{\alpha}{2}$ (так как распределение $N(0,1)$ является известным) и принимать или отвергать гипотезу в соответствии с ними.

\prp (Задача номер 3) Пусть $X_1,...,X_n$ — простая выборка, полученная из
абсолютно непрерывного распределения с плотностью $f$. Найти: 
\begin{itemize}
\item  Функцию плотности совместного распределения вариационного ряда $X_{(1)},...,X_{(n)}$.
\item Совместное распределение $X_{(1)}$ и $X_{(n)}$ при условии, что $X_i$ имеет равномерное распределение на отрезке $[a,b]$. Вычислить также их математические
ожидания, дисперсии и корреляцию.
\end{itemize}

\textbf{Решение: }

\begin{itemize}

\item Функцию плотности рассмотрим для $n=2$, для $n \geq 2$ ее построение ничем не отличается: Пусть $x_1 < x_2$ и будем рассматривать интервалы, по которым могут распределиться все наши элементы ряда: ровно $i_{1}-1$ должны попасть в интервал до $x_1$, один из элементов в $[x_1, x_1 + dx]$, дальше ещё $i_2 - 1 - i_1$ элемент ровно между $x_1+dx$ и $x_2$, один элемент в $[x_2, x_2 + dx]$ и все оставшиеся дальше $x_2+dx$. Таким образом, получим вероятность такого события:

\begin{align*}
C_n^{i_1 -1} F^{i_1 -1}(x_1)(n - i_1 + 1) f(x_1)dxC^{i_2 - 1 -i_1}_{n-i_1}(F(x_2)- \\F(x_1+dx))^{i_2 - 1 - i_1}(n-i_2+1)f(x_2)dx(1 - F(x_2))^{n-i_2}
\end{align*}

Просто берем предел при $dx \rightarrow 0$ и получаем функцию распределения для $X{(i_1)}$ и $X_{(i_2)}$. Дальше аналогично для произвольного $n$.


\item Теперь рассмотрим совместное распределение для двух элементов $i_1 = 1, i_2 = n$, по формуле, выведенной выше и при условии $X_i$ - равномерна на $[a,b]$

\begin{align*}
C_n^{0} F^{0}(x_1)(n) f(x_1)C^{n - 2}_{n-1}(F(x_2)- \\F(x_1))^{n - 2}(n)f(x_2)dx(1 - F(x_2))^{0}
\end{align*}

Где $F$ и $f$ - ФР и ФП для равномерного на $[a,b]$

Математическое ожидание для 

\end{itemize}

\prp

\prp

\prp Проанализируйте данные о возрасте и доходах по ссылке:
http://lib.stat.cmu.edu/DASL/Datafiles/montanadat.html

\textbf{Решение: } Воспользуемся аналогично задаче 1 гипотезой независимости и используем критерий $\chi^2$:

Ответ: 

\end{document}