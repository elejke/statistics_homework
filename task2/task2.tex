 \documentclass[12pt]{article}
\usepackage[T2A]{fontenc}
\usepackage[utf8]{inputenc}        % Кодировка входного документа;
                                    % при необходимости, вместо cp1251
                                    % можно указать cp866 (Alt-кодировка
                                    % DOS) или koi8-r.

\usepackage[english,russian]{babel} % Включение русификации, русских и
                                    % английских стилей и переносов
%%\usepackage{a4}
%%\usepackage{moreverb}
\usepackage{amsmath,amsfonts,amsthm,amssymb,amsbsy,amstext,amscd,amsxtra,multicol}
\usepackage{verbatim}
\usepackage{mathtools}
\usepackage{tikz} %Рисование автоматов
\usetikzlibrary{automata,positioning}
\usepackage{multicol} %Несколько колонок
\usepackage{graphicx}
\usepackage[colorlinks,urlcolor=blue]{hyperref}
\usepackage[stable]{footmisc}
%\usepackage{hyperref}

%% \voffset-5mm
%% \def\baselinestretch{1.44}
\renewcommand{\theequation}{\arabic{equation}}
\def\hm#1{#1\nobreak\discretionary{}{\hbox{$#1$}}{}}
\newtheorem{Lemma}{Лемма}
\theoremstyle{definiton}
%%\newtheorem{Def}{Определение}
\newtheorem{Claim}{Утверждение}
\newtheorem{Cor}{Следствие}
\newtheorem{Theorem}{Теорема}
\theoremstyle{definition}
\newtheorem{Example}{Пример}
\newtheorem*{known}{Теорема}
\def\proofname{Доказательство}
\def\solutionname{Решение}
\theoremstyle{definition}
\newtheorem{Def}{Определение}

%% \newenvironment{Example} % имя окружения
%% {\par\noindent{\bf Пример.}} % команды для \begin
%% {\hfill$\scriptstyle\qed$} % команды для \end






%\date{22 июня 2011 г.}
\let\leq\leqslant
\let\geq\geqslant
\def\MT{\mathrm{MT}}
%Обозначения ``ажуром''
\def\BB{\mathbb B}
\def\CC{\mathbb C}
\def\RR{\mathbb R}
\def\SS{\mathbb S}
\def\ZZ{\mathbb Z}
\def\NN{\mathbb N}
\def\FF{\mathbb F}
%греческие буквы
\let\epsilon\varepsilon
\let\es\emptyset
\let\eps\varepsilon
\let\al\alpha
\let\sg\sigma
\let\ga\gamma
\let\ph\varphi
\let\om\omega
\let\ld\lambda
\let\Ld\Lambda
\let\vk\varkappa
\let\Om\Omega
\def\abstractname{}

\def\R{{\cal R}}
\def\A{{\cal A}}
\def\B{{\cal B}}
\def\C{{\cal C}}
\def\D{{\cal D}}
\let\w\omega


%вероятность 
\newcommand{\Expect}{\mathsf{E}}
\newcommand{\MExpect}{\mathsf{M}}

%классы сложности
\def\REG{{\mathsf{REG}}}
\def\CFL{{\mathsf{CFL}}}
\newcounter{problem}
\newcounter{uproblem}
\newcounter{subproblem}
\def\pr{\medskip\noindent\stepcounter{problem}{\bf \theproblem .  }\setcounter{subproblem}{0} }
\def\prstar{\medskip\noindent\stepcounter{problem}{\bf $\theproblem^*$\negthickspace.  }\setcounter{subproblem}{0} }
\def\prpfrom[#1]{\medskip\noindent\stepcounter{problem}{\bf Задача \theproblem~(№#1 из задания).  }\setcounter{subproblem}{0} }
\def\prp{\medskip\noindent\stepcounter{problem}{\bf Задача \theproblem .  }\setcounter{subproblem}{0} }
\def\prpstar{\medskip\noindent\stepcounter{problem}{\bf Задача $\bf\theproblem^*$\negthickspace.  }\setcounter{subproblem}{0} }
\def\prdag{\medskip\noindent\stepcounter{problem}{\bf Задача $\theproblem^{^\dagger}$\negthickspace\,.  }\setcounter{subproblem}{0} }
\def\upr{\medskip\noindent\stepcounter{uproblem}{\bf Упражнение \theuproblem .  }\setcounter{subproblem}{0} }
%\def\prp{\vspace{5pt}\stepcounter{problem}{\bf Задача \theproblem .  } }
%\def\prs{\vspace{5pt}\stepcounter{problem}{\bf \theproblem .*   }
\def\prsub{\medskip\noindent\stepcounter{subproblem}{\rm \thesubproblem .  } }
\def\prsubstar{\medskip\noindent\stepcounter{subproblem}{\rm $\thesubproblem^*$\negthickspace.  } }
%прочее
\def\quotient{\backslash\negthickspace\sim}
\begin{document}

\centerline{\LARGE Домашнее задание 2, Новиков Герман, 277}

\medskip

Все вычисления вместе с кодом и комментариями находятся в файле на $\href{https://github.com/elejke/statistics_homework/blob/master/task2/code/task2_code.ipynb}{\textbf{github}}$

\prp (Задача нормер 9). Цифры $0, 1, 2, . . . , 9$ среди $800$ первых десятичных знаков числа $\pi$ появляются $74, 92, 83, 79, 80, 73, 77, 75, 76, 91$ раз соответственно.

Проверить гипотезу о согласии данных с законом равномерного распределения.


\textbf{Решение: } Для проверки гипотезы о равномерности распределении данных воспользуемся критерием $\chi^2$:

Получаемое значение $T_{\chi^2}$ отвечает нашему предположению для $\alpha = 0.05$, соотвественно, принимаем гипотезу о равномерном распределении.

\prp $data = [50 numbers]$

Используя Пирсоновский хи-квадрат тест, проверить гипотезу о том, что эти числа распределены согласно равномерному распределению на отрезке [0, 1]. 

\textbf{Решение: } Аналогично предыдущей задаче, однако в этот раз воспользуемся $p-value$.

Полученное значение $p_{value} = 0.23$ показывает, что с достаточной вероятностью данные распределены равномерно на $[0,1]$, что подтверждает нашу гипотезу.

\prp (Задача номер 10) При эпидемии гриппа из 200 контролируемых людей
однократное заболевание наблюдалось у 181 человека, а дважды болели гриппом
9 человек. Правдоподобна ли гипотеза о том, что в течение эпидемии гриппа
число заболеваний отдельного человека представляет собой случайную величину,
подчиняющуюся биномиальному распределению с числом испытаний $n = 2$?

\textbf{Решение: } Изначально, мы не знаем распределение вероятностей (то есть не знаем параметр $p$), строим правдоподобие для него и находим предполагаемое значение. Дальше используем критерий $\chi^2$.

Полученное $p_{value} = 3.18e^{-29}$ есть почти нулевая величина, что означает, что мы получили очень редкое событие и, в соответствии с этим, мы отвергаем нашу гипотезу о нормальном распределении.

\prp Пусть некоторая статистика, построенная по простой выборке, в условиях истинности гипотезы $H_0$ имеет абсолютно непрерывное распределение. 
Рассматривая $p-value$ как случайную величину, найти ее распределение.

\textbf{Решение: } В предположении, что нулевая гипотеза $H_0$ справедлива, рассматриваемая нами статистика $T(X)$ имеет распределение $F(t)$. Рассматриваем распределение $p-value$ (учитывая, что статистика имеет абсолютно непрерывное распределение), фактически - $p-value$ - случайная величина: $\xi = 1 - F(T(X))$, но тогда:



$$\Prob(\xi < t) = \Prob(1 - F(T(X)) < t) = 1 - \Prob(F(T(X)) > 1 - t) = $$
$$1 - \Prob(T(X) > F^{-1}(1-t)) = 1 - F(F^{-1}(1-t)) = 1 - (1-t) = t$$

То есть мы получаем, что $p-value$ для любого распределения есть равномерно распределенная на $[0,1]$ случайная величина.

\prp (Задача номер 12). При снятии показаний измерительного прибора
десятые доли деления шкалы прибора оцениваются "на глаз" \newline наблюдателем.
Количества цифр $0, 1, 2, ..., 9$, записанных наблюдателем в качестве десятых долей 1 при 100 независимых измерениях, равны $5, 8, 6, 12, 14, 18, 11, 6, 13, 7$ соответственно.

Проверить гипотезы о согласии данных с законом равномерного распределения и
с законом нормального распределения. Для ответа на вопрос можно сравнить
значения p-value для обеих гипотез.

\textbf{Решение: } С семинара и в Ивченко-Медведеве выведены формулы для $\widehat{\theta_1}$ и $\widehat{\theta_2}$, остается посчитать их численно и подставить в $T_{chi^2}$, для которого у нас будет $N-1-2$ степеней свободы в силу независимости $\theta_1$ и $\theta_2$.

\end{document}