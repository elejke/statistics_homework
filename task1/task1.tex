 \documentclass[12pt]{article}
\usepackage[T2A]{fontenc}
\usepackage[utf8]{inputenc}        % Кодировка входного документа;
                                    % при необходимости, вместо cp1251
                                    % можно указать cp866 (Alt-кодировка
                                    % DOS) или koi8-r.

\usepackage[english,russian]{babel} % Включение русификации, русских и
                                    % английских стилей и переносов
%%\usepackage{a4}
%%\usepackage{moreverb}
\usepackage{amsmath,amsfonts,amsthm,amssymb,amsbsy,amstext,amscd,amsxtra,multicol}
\usepackage{verbatim}
\usepackage{mathtools}
\usepackage{tikz} %Рисование автоматов
\usetikzlibrary{automata,positioning}
\usepackage{multicol} %Несколько колонок
\usepackage{graphicx}
\usepackage[colorlinks,urlcolor=blue]{hyperref}
\usepackage[stable]{footmisc}

%% \voffset-5mm
%% \def\baselinestretch{1.44}
\renewcommand{\theequation}{\arabic{equation}}
\def\hm#1{#1\nobreak\discretionary{}{\hbox{$#1$}}{}}
\newtheorem{Lemma}{Лемма}
\theoremstyle{definiton}
%%\newtheorem{Def}{Определение}
\newtheorem{Claim}{Утверждение}
\newtheorem{Cor}{Следствие}
\newtheorem{Theorem}{Теорема}
\theoremstyle{definition}
\newtheorem{Example}{Пример}
\newtheorem*{known}{Теорема}
\def\proofname{Доказательство}
\def\solutionname{Решение}
\theoremstyle{definition}
\newtheorem{Def}{Определение}

%% \newenvironment{Example} % имя окружения
%% {\par\noindent{\bf Пример.}} % команды для \begin
%% {\hfill$\scriptstyle\qed$} % команды для \end






%\date{22 июня 2011 г.}
\let\leq\leqslant
\let\geq\geqslant
\def\MT{\mathrm{MT}}
%Обозначения ``ажуром''
\def\BB{\mathbb B}
\def\CC{\mathbb C}
\def\RR{\mathbb R}
\def\SS{\mathbb S}
\def\ZZ{\mathbb Z}
\def\NN{\mathbb N}
\def\FF{\mathbb F}
%греческие буквы
\let\epsilon\varepsilon
\let\es\emptyset
\let\eps\varepsilon
\let\al\alpha
\let\sg\sigma
\let\ga\gamma
\let\ph\varphi
\let\om\omega
\let\ld\lambda
\let\Ld\Lambda
\let\vk\varkappa
\let\Om\Omega
\def\abstractname{}

\def\R{{\cal R}}
\def\A{{\cal A}}
\def\B{{\cal B}}
\def\C{{\cal C}}
\def\D{{\cal D}}
\let\w\omega


%вероятность 
\newcommand{\Expect}{\mathsf{E}}
\newcommand{\MExpect}{\mathsf{M}}

%классы сложности
\def\REG{{\mathsf{REG}}}
\def\CFL{{\mathsf{CFL}}}
\newcounter{problem}
\newcounter{uproblem}
\newcounter{subproblem}
\def\pr{\medskip\noindent\stepcounter{problem}{\bf \theproblem .  }\setcounter{subproblem}{0} }
\def\prstar{\medskip\noindent\stepcounter{problem}{\bf $\theproblem^*$\negthickspace.  }\setcounter{subproblem}{0} }
\def\prpfrom[#1]{\medskip\noindent\stepcounter{problem}{\bf Задача \theproblem~(№#1 из задания).  }\setcounter{subproblem}{0} }
\def\prp{\medskip\noindent\stepcounter{problem}{\bf Задача \theproblem .  }\setcounter{subproblem}{0} }
\def\prpstar{\medskip\noindent\stepcounter{problem}{\bf Задача $\bf\theproblem^*$\negthickspace.  }\setcounter{subproblem}{0} }
\def\prdag{\medskip\noindent\stepcounter{problem}{\bf Задача $\theproblem^{^\dagger}$\negthickspace\,.  }\setcounter{subproblem}{0} }
\def\upr{\medskip\noindent\stepcounter{uproblem}{\bf Упражнение \theuproblem .  }\setcounter{subproblem}{0} }
%\def\prp{\vspace{5pt}\stepcounter{problem}{\bf Задача \theproblem .  } }
%\def\prs{\vspace{5pt}\stepcounter{problem}{\bf \theproblem .*   }
\def\prsub{\medskip\noindent\stepcounter{subproblem}{\rm \thesubproblem .  } }
\def\prsubstar{\medskip\noindent\stepcounter{subproblem}{\rm $\thesubproblem^*$\negthickspace.  } }
%прочее
\def\quotient{\backslash\negthickspace\sim}
\begin{document}

\centerline{\LARGE Домашнее задание 1, Новиков Герман, 277}

\medskip

\prp Пусть $\hat{F}_n$ — эмпирическая функция распределения, построенная по
простой выборке длины $n$, с неизвестной функцией распределения $F$.

\begin{itemize}
\item Какое распределение имеет случайная величина $n \hat{F}_n$?
\item Используя оценку $\hat{F}_n(x)$ и вариант центральной предельной теоремы, построить приближенный (в виду предельности теоремы) интервал, содержащий значение
$F(x)$ с заданной вероятностью $\alpha$.
\end{itemize}

\textbf{Решение: }

\begin{itemize}
\item По определению: $$n \hat{F}_n(x) = \sum\limits_{i=1}^n I(X_i < x)$$. Но тогда, для произвольного $x$: $n \hat{F}_n(x)$ есть сумма $n$ независимых Бернуллевских случайных велечин с $p = F(x)$, а следственно $$n \hat{F}_n(x) \sim Bin(n, F(x))$$


\item 

$$\sqrt{n}(\hat{F}_n(x) - F(x)) = \left(\frac{\sum\limits_{i=1}^n I(X_i < x) - nF(x)}{\sqrt{n}}\right) = $$ 

$$= \left(\frac{\sum\limits_{i=1}^n I(X_i < x) - n\Expect(I(X_1 < x))}{\sqrt{n}}\right) \xrightarrow{CLT} N(0,\Variance(I(X_1 < x)))$$

Где $\Variance(I(X_1 < x)) = F(x)(1 - F(x))$

Таким образом, получаем ассимптотическую оценку на распределение разности между эмперической и реальной функцией распределения, взвешенной на $\sqrt{n}$. Положим, что выборка достаточно большая и можно считать, что равенство не ассимптотическое. Тогда мы просто берем двухсторонний квантиль: $$x^{(1)}_{\frac{\alpha}{\sqrt{n}}} = G^{-1}\left(\frac{\alpha}{2\sqrt{n}}\right), x^{(2)}_{\frac{\alpha}{\sqrt{n}}} = G^{-1}\left(1 -\frac{\alpha}{2\sqrt{n}}\right)$$

Где $G(x)$ - та самая, полученная нами предельная функция распределения.

\end{itemize}


\begin{Theorem}
(Неравенство Дворецкого-Кифера-Вольфовитца).
\end{Theorem}


Пусть $X_1,...,X_n \sim F$. Тогда для любого $\epsilon > 0$:

$$\Prob\left(\sup\limits_{x \in \mathbb{R}}|\hat{F}_n(x) - F(x)| \geq \epsilon\right) \leq 2 \exp (-2n\epsilon^2)$$

\prp Построить на основании неравенства Дворецкого-Кифера-Вольфовитца
неасимптотический вариант критерия согласия Колмогорова. Указать критическую
область при заданном уровне значимости.

\textbf{Решение: } критерий согласия Колмогорова:

$$\Prob\left(\sqrt{n}\sup\limits_{x \in \mathbb{R}}|\hat{F}_n(x) - F(x)| \leq \epsilon\right) =  \Prob\left(\sup\limits_{x \in \mathbb{R}}|\hat{F}_n(x) - F(x)| \leq \frac{\epsilon}{\sqrt{n}}\right) \leq 1 - 2 \exp (-2\epsilon^2)$$

Задаём $\alpha$ - уровень значимости: $\sqrt{n} \sup\limits_{x \in \mathbb{R}}|\hat{F}_n(x) - F(x)| \geq t_{\alpha}$

\medskip

В нашем случае $K(\epsilon) = 1 - 2\exp(-2\epsilon^2)$

\medskip

Хотим: $\alpha = 1 - K(\epsilon)$, следственно выбираем $$t_{\alpha} = \epsilon = K^{-1}(1-\alpha) = \sqrt{\frac{\ln(2) - ln(\alpha)}{2}}$$

Таким образом отвергаем гипотезу $H_0$ при: $$\sqrt{n} D_n \geq \sqrt{\frac{\ln(2) - ln(\alpha)}{2}}$$



\prp Доказать, что $$\sup\limits_{x \in \mathbb{R}} |\hat{F}(x) - F(x)| \xrightarrow{a.e.} 0,$$ где предел берется по $n$ - числу элементов выборки.

\textbf{Доказательство(можно не комментировать, это доказательство из книги, свое я обещаю придумать): } Пусть $F(x)$ непрерывна. Фиксируем $\epsilon = 1/N$. Так как $\forall x F(x) \in [0,1]$ и непрерывна, то существует набор чисел $\{x_i\}_{i=1}^N \subset \mathbb{R}\cup\{-\infty, +\infty\}$ такой, что на каждом из отрезков $[x_i, x_i+1]$ функция $F(x)$ растёт на $\frac1N$, где $N$ - произвольное целое. Тогда $\forall x \in [x_i, x_{i+1}]$ выполнено:

$$\hat{F}(x) - F(x) \leq \hat{F}(x_{i+1}) - F(x_i) = \hat{F}(x_{i+1}) - F(x_{i+1}) + \epsilon$$

$$\hat{F}(x) - F(x) \geq \hat{F}(x_{i}) - F(x_i+1) = \hat{F}(x_{i}) - F(x_{i}) - \epsilon$$


Обозначим $A_k$ - множество элементарных событий $\omega$, на которых $\hat{F}(x_k)\rightarrow F(x_k)$. Мера этого множества $P(A_k) = 1$ в силу УЗБЧ. Но тогда $\forall \omega \in A = \cup_{k=0,N} A_k$ существует $n(\omega)$: что $\forall n \geq n(\omega)$ выполнено

$$|\hat{F}(x_k) - F(x_k)| < \epsilon, k = 0,1,...,N.$$

Но тогда вместе с предыдущим получаем требуемое утверждение:

$$\sup\limits_z |\hat{F}(x) - F(x)| < 2\epsilon.$$

Для произвольной функции аналогично. Только нужно более аккуратно поступать с разрывами.


\prp Построить plug-in оценки для математического ожидания и дисперсии.
Являются ли эти оценки несмещенными?

\textbf{Решение: } 

\begin{itemize}
\item Для математического ожидания $\overline{x} = \hat{\Expect}(X) = \frac1n\sum\limits^n_{i=1} X_i$. Она является несмещенной, т.к. $\Expect\hat{\Expect}(X) = \Expect\frac1n\sum\limits^n_{i=1} X_i = \frac1n\sum\limits^n_{i=1} \Expect X_i = \Expect X$, где $X \sim F(x)$
\item Для дисперсии: $S^2 = \frac1n\sum\limits^n_{i=1} (X_i - \overline{x})^2$. Она не является несмещенной: 

$S^2 = \frac1n\sum\limits^n_{i=1} (X_i - \overline{x})^2 =
\frac1n\sum\limits^n_{i=1}(X_i^2 - 2X_i\overline{x} + \overline{x}^2) = \frac1n\sum\limits^n_{i=1}  X_i^2 - 2\overline{x}(\frac1n\sum\limits^n_{i=1} X_i) + \frac1n\sum\limits^n_{i=1} X_i^2 = \frac1{n}\sum\limits^n_{i=1} X_i^2 - \overline{x}^2  = \overline{x^2} - \overline{x}^2$

Тогда получим, что $\Expect S^2 = \Expect(\overline{x^2} - \overline{x}^2) = ... = \sigma^2 - \frac{\sigma^2}{n} = \sigma^2(1 - \frac1n)$. То есть не является несмещенной, однако просто нормируется до несмещенной, домножением на $\frac{n}{n-1}$.


\end{itemize}


\end{document}